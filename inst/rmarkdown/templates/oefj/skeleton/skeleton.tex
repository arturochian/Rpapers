\documentclass[article]{jss}

\author{
Primer\_nombre Apellidos\\FCE UNMSM \And Segundo Autor\\FCE UNMSM
}
\title{Un análisis a la banca con el paquete \pkg{BCRP}}
\Keywords{Minsky, banca, \proglang{BCRP}}

\Abstract{
Para el abstract, por favor usar inglés y español.
}

\Plainauthor{Primer\_nombre Apellidos, Segundo Autor}
\Plainkeywords{Minsky, banca, BCRP}

%% publication information
%% \Volume{50}
%% \Issue{9}
%% \Month{June}
%% \Year{2012}
%% \Submitdate{2012-06-04}
%% \Acceptdate{2012-06-04}

\Address{
    Primer\_nombre Apellidos\\
  FCE UNMSM\\
  First line Second line\\
  E-mail: \href{mailto:name@company.com}{name@company.com}\\
  URL: \url{http://rstudio.com}
    }

\usepackage{amsmath}

\begin{document}

\section{Introduction}\label{introduction}

This template demonstrates some of the basic latex you'll need to know
to create a JSS article.

\subsection{Code formatting}\label{code-formatting}

Don't use markdown, instead use the more precise latex commands:

\begin{itemize}
\itemsep1pt\parskip0pt\parsep0pt
\item
  \proglang{Java}
\item
  \pkg{plyr}
\item
  \code{print("abc")}
\end{itemize}

\section{R code}\label{r-code}

Can be inserted in regular R markdown blocks.

\begin{CodeChunk}
\begin{CodeInput}
x <- 1:10
x
\end{CodeInput}
\begin{CodeOutput}
 [1]  1  2  3  4  5  6  7  8  9 10
\end{CodeOutput}
\end{CodeChunk}

\end{document}

